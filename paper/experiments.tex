\section{Experimental Evidence}
Experimental evidence for quantum teleportation in quantum communications.
\begin{mybox}{Bullet points}
    \begin{itemize}
        \item Quantum teleportation has experimental evidence
        \item Experimental hurdles
    \end{itemize}
\end{mybox}
\subsection{Satellite Based}

%1400 km \cite{Ren:2017}
%\begin{mybox}{Bullet points}
%    \begin{itemize}
%        \item Protocol used
%        \item distance
%        \item technical difficulties and innovations
%        \item what does this mean for quantum communications?
%    \end{itemize}
%\end{mybox}

The qubit system used in this experiment was based on the polarization of a photon, represented as:  
\begin{equation}
    \ket{\chi}_1 = \alpha\ket{H}_1 + \beta\ket{V}_1,
\end{equation}
where \(\alpha\) and \(\beta\) are complex numbers satisfying \(|\alpha|^2 + |\beta|^2 = 1\), and \(\ket{H}\) and \(\ket{V}\) denote 
horizontal and vertical polarization states, respectively.  

At a ground station, two entangled photon pairs were prepared. One photon from each pair was transmitted to a satellite through a 
130-mm-diameter telescope, designed with narrow beam divergence and high-precision tracking systems to counteract atmospheric turbulence. 
The entangled pair of photons is described by one of the Bell states:
\begin{equation}
    \ket{\psi^{+}}_{23} = \frac{1}{\sqrt{2}} (\ket{H}_2\ket{H}_3 + \ket{V}_2\ket{V}_3),
\end{equation}
where photon 2 was retained at the ground station, and photon 3 was sent to the satellite.

A joint measurement, known as a Bell-state measurement, was performed on the photon to be teleported (photon 1) and photon 2 from the entangled pair. 
This measurement projected the two photons into one of the Bell states:
\begin{equation}
    \ket{\psi^{\pm}}_{12} = \frac{1}{\sqrt{2}} (\ket{H}_1\ket{H}_2 \pm \ket{V}_1\ket{V}_2).
\end{equation}

The outcome of the measurement was then transmitted classically to the satellite, where it influenced the state of photon 3. If the measured state was \(\ket{\psi^{+}}_{12}\), photon 3 adopted the original state of photon 1. If the measured state was \(\ket{\psi^{-}}_{12}\), photon 3 carried the original state of photon 1, but with a \(\pi\)-phase shift.  

This experiment demonstrated the successful teleportation of a photon's quantum state over distances of up to 1400 kilometers, from a ground station to a satellite in low-Earth orbit. The primary technical challenges included atmospheric turbulence, which caused beam wandering and broadening, resulting in significant signal losses. These challenges were addressed through the use of a narrow beam divergence, a high-precision telescope, and an advanced acquiring, pointing, and tracking (APT) system.

In the future, this method could be used to create a global quantum internet, connecting quantum computers across the globe and introducing secure cryptography protocols. 
This also raises the possibility of letting several quantum processors work together on a single problem, enhancing the efficacy of quantum computers.

\subsection{Fibre Network Based}
100 km \cite{Takesue:2015}. Metropolitan \cite{Valivarthi:2016}
\begin{mybox}{Bullet points}
    \begin{itemize}
        \item Protocol used
        \item distance
        \item technical difficulties and innovations
        \item what does this mean for quantum communications?
    \end{itemize}
\end{mybox} 
