\section{Experimental Evidence}
Experimental evidence for quantum teleportation in quantum communications.
\begin{mybox}{Bullet points}
    \begin{itemize}
        \item Quantum teleportation has experimental evidence
        \item Experimental hurdles
    \end{itemize}
\end{mybox}
\subsection{Satellite Based}

1400 km \cite{Ren:2017}
\begin{mybox}{Bullet points}
    \begin{itemize}
        \item Protocol used
        \item distance
        \item technical difficulties and innovations
        \item what does this mean for quantum communications?
    \end{itemize}
\end{mybox}

The qubit system used in this experiment is based on the polarization of a photon with a state represented as $\ket{\chi}_1 = \alpha\ket{H}_1 + \beta\ket{V}_1$, with complex numbers
satisfying $\vert\alpha\vert² + \vert\beta\vert² = 1$ and $\ket{V}$ and $\ket{H}$ represents vertical and horizontal polarization respectively.
 At a ground station, two entangled photon pairs were prepared and one photon from each pair was
transmitted to a Satellite through a 130-mm diameter telescope with narrow beam divergence and high-precision tracking systmes to counteract the atmospheric turbulence.
The entangled pair of photons can be written as one of the Bell states,
\begin{equation}
    \ket{\psi^{+}}_{23} = (\ket{H}_2\ket{H}_3 + \ket{V}_2\ket{V}_3)/\sqrt{2},
\end{equation}
where photon 2 is the photon kept at the ground station and photon 3 was sent to the Satellite.

A joint measurement was one on the photon that was to be teleported and one of the entangled photons. This projects these photons into one of the Bell states.
The result from this measurment was then classically sent to the Satellite. This measurement forces the entangled pair into a new state. If this state is 
$\ket{\psi^{+}}_{12} = (\ket{H}_1\ket{H}_2 + \ket{V}_1\ket{V}_2)/\sqrt{2}$ the photon sent to the Satellite (photon 3) carries the state that photon 1 was in originally
and if the measured state is $\ket{\psi^{-}}_{12} = (\ket{H}_1\ket{H}_2 - \ket{V}_1\ket{V}_2)/\sqrt{2}$, the state of photon 3 is equivalent of the state of photon 1 shifted $\pi$.

In this paper it was shown that a photon's quantum state could be teleported over a distance up to 1400 kilometers, from the ground station to the Satellite in low-Earth orbit. The biggest technical 
difficulties with this approach is the atmospheric turbulence causing the beam to wander and broaden, leading to significant signal losses. This was dealt with by using a 
narrow beam divergence from a high precision telescope and an advanced acquiring, pointing and tracking system.

In the future, this method could be used to create a global quantum internet, connecting quantum computers across the globe and introducing secure cryptography protocols. 
This also raises the possibility of letting several quantum processors work together on a single problem, enhancing the efficacy of quantum computers.

\subsection{Fibre Network Based}
100 km \cite{Takesue:2015}. Metropolitan \cite{Valivarthi:2016}
\begin{mybox}{Bullet points}
    \begin{itemize}
        \item Protocol used
        \item distance
        \item technical difficulties and innovations
        \item what does this mean for quantum communications?
    \end{itemize}
\end{mybox} 
