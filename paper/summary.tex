\section{Summary \& Conclusion}
Quantum teleportation stands as a cornerstone of quantum communication, providing a method to transfer quantum states across distant locations without the physical movement of particles. This report examined the theoretical underpinnings, technological advances, and experimental implementations of quantum teleportation. It began by detailing the foundational concepts, including Einstein-Podolsky-Rosen (EPR) pairs, Bell states, and the quantum teleportation protocol. Using entanglement and classical communication channels, teleportation allows the exact state of a quantum particle to be replicated remotely, ensuring the fidelity and security of the process.

The report explored teleportation of complex quantum systems, focusing on the use of qudits in high-dimensional states to enhance information density and practical applications. Challenges such as creating high-quality entanglement and performing accurate Bell measurements were addressed, with experimental realizations showcasing promising advancements like the use of auxiliary particles and quantum Fourier transformations.

To enable long-distance quantum teleportation, technologies such as quantum repeaters and quantum memory were highlighted. These components mitigate signal loss and enhance scalability by enabling entanglement swapping and reliable storage of quantum states. Experimental achievements were discussed, showcasing satellite-based and fiber-network-based teleportation. Satellite-based teleportation demonstrated quantum state transfer over \SI{1400}{\kilo\meter} with high fidelity, overcoming atmospheric and tracking challenges. Fiber-based teleportation achieved urban network integration and long-distance communication through advancements in time-bin encoding and efficient photon detectors.

In conclusion, quantum teleportation represents a pivotal step toward realizing a global quantum internet, capable of secure communication and distributed quantum computing. Current experimental successes provide a strong foundation, but continued advancements in scalability, error mitigation, and infrastructure integration are essential. These efforts promise to revolutionize communication, computation, and fundamental quantum research, propelling the field into an era of unprecedented possibilities.

