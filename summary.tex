\documentclass[paper=a4, fontsize=11pt]{scrartcl} % A4 paper and 11pt font size

\usepackage[T1]{fontenc} % Use 8-bit encoding that has 256 glyphs
% \usepackage{fourier} % Use the Adobe Utopia font for the document - comment this line to return to the LaTeX default
\usepackage[english]{babel} % English language/hyphenation
\usepackage{amsmath,amsfonts,amsthm} % Math packages
\usepackage{lipsum} % Used for inserting dummy 'Lorem ipsum' text into the template

\usepackage{bbold}
\usepackage{caption}
\usepackage{subcaption}
\usepackage{graphicx}
\usepackage[section]{placeins}
\usepackage{float}
\usepackage[bottom=3cm, left=1cm,right=1cm]{geometry}
\usepackage{blindtext} %for enumarations
\usepackage{physics}

\usepackage[]{hyperref}  %link collor
% \usepackage{bibtex}
\usepackage{gensymb}

% talbe layout to the right
% \usepackage[labelfont=bf]{caption}
% \captionsetup[table]{labelsep=space,justification=raggedright,singlelinecheck=off}
% \captionsetup[figure]{labelsep=quad}

\usepackage{sectsty} % Allows customizing section commands
\allsectionsfont{\centering \normalfont} % Make all sections centered, the default font and small caps

\usepackage{fancyhdr} % Custom headers and footers
\pagestyle{fancyplain} % Makes all pages in the document conform to the custom headers and footers
\fancyhead{} % No page header - if you want one, create it in the same way as the footers below
\fancyfoot[L]{} % Empty left footer
\fancyfoot[C]{} % Empty center footer
\fancyfoot[R]{\thepage} % Page numbering for right footer
\renewcommand{\headrulewidth}{0pt} % Remove header underlines
\renewcommand{\footrulewidth}{0pt} % Remove footer underlines
\setlength{\headheight}{13.6pt} % Customize the height of the header

% \numberwithin{equation}{section} % Number equations within sections (i.e. 1.1, 1.2, 2.1, 2.2 instead of 1, 2, 3, 4)
\numberwithin{figure}{section} % Number figures within sections (i.e. 1.1, 1.2, 2.1, 2.2 instead of 1, 2, 3, 4)
\numberwithin{table}{section} % Number tables within sections (i.e. 1.1, 1.2, 2.1, 2.2 instead of 1, 2, 3, 4)

\setlength\parindent{0pt} % Removes all indentation from paragraphs - comment this line for an assignment with lots of text


\setlength\parskip{4pt}

\newcommand{\horrule}[1]{\rule{\linewidth}{#1}} % Create horizontal rule command with 1 argument of height

\title{	
	\normalfont \normalsize 
	\textsc{} \ [25pt] % Your university, school and/or department name(s)
	\horrule{0.5pt} \[0.4cm] % Thin top horizontal rule
	\huge FYST85 - HANDIN 1 \ % The assignment title
	\horrule{2pt} \[0.5cm] % Thick bottom horizontal rule
}
\author{Lukas Nord}
\hyphenchar\font=-1
\sloppy

\begin{document}

\section*{Summary of Progress in Quantum Teleportation}

This paper provides a comprehensive review of the advancements in \textbf{quantum teleportation}, a cornerstone protocol in quantum information science, with a focus on developments since 2015. Quantum teleportation allows the transfer of an unknown quantum state from one location to another without the physical transmission of the particle itself. This process requires two channels: a \textbf{quantum channel} (via entanglement) and a \textbf{classical channel} (to communicate measurement outcomes).

Initially, quantum teleportation was a theoretical concept, but it has now evolved into a practical tool with significant implications for \textbf{quantum communication} and \textbf{quantum computing}.

\subsection*{Key Highlights}
\begin{enumerate}
    \item \textbf{Theoretical Foundations:}
    \begin{itemize}
        \item The protocol requires pre-shared quantum entanglement and classical communication to achieve state transfer.
        \item Recent work has clarified the \textbf{nonclassical nature} of teleportation, demonstrating that quantum entanglement enables fidelity beyond classical limits.
        \item Variants such as \textbf{port-based teleportation} and \textbf{quantum entanglement swapping} have broadened theoretical understanding and applications.
    \end{itemize}
    
    \item \textbf{Experimental Developments:}
    \begin{itemize}
        \item \textbf{Complex Quantum States:} Researchers have successfully teleported high-dimensional quantum states and multiple degrees of freedom (DoFs), such as polarization and orbital angular momentum. These advancements aim to encode more information in single photons, enhancing efficiency.
        \item \textbf{Long-Distance Teleportation:} Experiments have demonstrated teleportation over significant distances using:
        \begin{itemize}
            \item \textbf{Metropolitan fiber networks:} Teleportation has been achieved across city-scale networks with high fidelity.
            \item \textbf{Free-space satellite links:} Ground-to-satellite teleportation over 1,400 km has set records for long-distance quantum communication.
            \item \textbf{Integrated photonic chips:} These enable compact and stable platforms for teleportation with potential for scalable quantum networks.
        \end{itemize}
    \end{itemize}

    \item \textbf{Applications:}
    \begin{itemize}
        \item \textbf{Quantum Communication:}
        \begin{itemize}
            \item Teleportation extends the reach of quantum networks by overcoming the distance limitations of direct transmission using \textbf{quantum repeaters}.
            \item Satellite-based teleportation demonstrates feasibility for a global quantum internet.
        \end{itemize}
        \item \textbf{Quantum Computing:}
        \begin{itemize}
            \item \textbf{Gate teleportation} enables interaction between spatially separated qubits, forming the basis for distributed quantum computing.
            \item Fault-tolerant quantum computing benefits from teleportation-based encoding and error correction.
        \end{itemize}
    \end{itemize}
\end{enumerate}

\subsection*{Outlook}
The paper identifies several challenges and opportunities:
\begin{itemize}
    \item \textbf{Challenges:}
    \begin{itemize}
        \item Improving entanglement generation and Bell-state measurement (BSM) efficiency.
        \item Scaling quantum teleportation for practical, high-speed quantum networks.
    \end{itemize}
    \item \textbf{Opportunities:}
    \begin{itemize}
        \item Advances in deterministic photon sources and nonlinear optics could enhance teleportation success rates.
        \item Hybrid systems combining photonic, atomic, and solid-state technologies promise robust quantum networks with diverse applications.
    \end{itemize}
\end{itemize}

Overall, quantum teleportation is positioned as a vital technology for realizing the vision of a \textbf{quantum internet} and for enabling distributed, large-scale quantum computing.


\end{document}
